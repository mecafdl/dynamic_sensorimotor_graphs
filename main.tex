%\documentclass[letterpaper, 10 pt, conference]{ieeeconf}
%\usepackage{filecontents,lipsum}
%\usepackage[noadjust]{cite}
%%\begin{filecontents*}{references.bib}
%%@article{Khoe:1994:CML:2288694.2294265,
%%    author = {Khoe, G. -D.},
%%    title = {Coherent multicarrier lightwave technology for flexible capacity networks},
%%    journal = {Comm. Mag.},
%%    issue_date = {March 1994},
%%    volume = {32},
%%    number = {3},
%%    month = mar,
%%    year = {1994},
%%    issn = {0163-6804},
%%    pages = {22--33},
%%    numpages = {12},
%%    url = {http://dx.doi.org/10.1109/35.267438},
%%    doi = {10.1109/35.267438},
%%    acmid = {2294265},
%%    publisher = {IEEE Press},
%%    address = {Piscataway, NJ, USA},
%%}
%%\end{filecontents*}
%\title{This document}
%\author{This author}

%%%%%%%%%%%%%%%%%%%%%%%%%%%%%%%%%%%%%%%%%%%%%%%%%%%%%%%%%%%%%%%%%%%%%%%%%%%%%%%%
%2345678901234567890123456789012345678901234567890123456789012345678901234567890
%        1         2         3         4         5         6         7         8

\documentclass[letterpaper, 10 pt, conference]{ieeeconf}  % Comment this line out if you need a4paper

%\documentclass[a4paper, 10pt, conference]{ieeeconf}      % Use this line for a4 paper

\IEEEoverridecommandlockouts                              % This command is only needed if 
                                                          % you want to use the \thanks command

\overrideIEEEmargins                                      % Needed to meet printer requirements.

% See the \addtolength command later in the file to balance the column lengths
% on the last page of the document

% The following packages can be found on http:\\www.ctan.org
%\usepackage{graphics} % for pdf, bitmapped graphics files
%\usepackage{epsfig} % for postscript graphics files
%\usepackage{mathptmx} % assumes new font selection scheme installed
%\usepackage{times} % assumes new font selection scheme installed
%\usepackage{amsmath} % assumes amsmath package installed
%\usepackage{amssymb}  % assumes amsmath package installed
%\usepackage{filecontents,lipsum}
%\usepackage[noadjust]{cite}
%\usepackage{bm}
%\usepackage{tikz}
%\usepackage{graphicx}
%\usepackage{caption}
%\usepackage{subcaption}
%\usepackage{multicol}
%\usepackage{url}
%\usepackage{geometry}
%\usepackage{mathtools}
%\usepackage[utf8]{inputenc}
%\usepackage[english]{babel}
%\usepackage{algorithm}
%\usepackage[]{algpseudocode}


\usepackage{afterpage}
\usepackage{algorithm}
\usepackage[]{algpseudocode}
\usepackage{amsmath}
\usepackage{amssymb}  % assumes amsmath package installed
\usepackage{arydshln}
\usepackage[english]{babel}
\usepackage{bm}
\usepackage{caption}
\usepackage[T1]{fontenc}
\usepackage{graphicx}
\usepackage[utf8]{inputenc}
\usepackage{multicol}
\usepackage[T1]{xcolor}
\usepackage{soul}
\usepackage{subfig}
\usepackage{tikz}
\usepackage{url}
\usepackage[backend=biber,style=ieee,sorting=none]{biblatex}
\addbibresource{bib/references.bib}

\newcommand\blankpage{%
    \null
    \thispagestyle{empty}%
    \addtocounter{page}{-1}%
    \newpage}



\newcommand*{\important}[1]{\textcolor{red}{\danger~\textbf{IMPORTANT:~}} \textcolor{red}{#1}}

\newcommand*{\pending}[1]{\textcolor{blue}{$\bigstar$~\textbf{PENDING~#1}}}

\newcommand\mybox[2][]{\tikz[overlay]\node[fill=blue!100,inner sep=4pt, anchor=text, rectangle, rounded corners=1mm,#1] {#2};\phantom{#2}}

\newcommand{\TODO}{\mybox[fill=yellow]{\textcolor{blue}{\Large \textbf{TODO}}}}

\newtheorem{prop}{Proposition}

\title{\LARGE \bf
Dynamic sensorimotor graphs
}


\author{Some Guy, Nice Dude}

\begin{document}

\maketitle

\begin{abstract}
Regularities present in the sensorimotor signals of a robotic agent can reflect its embodiment, as well as the associations resulting from the active control policy. In this work, we analyze the functional connectivity of the sensorimotor signals based on pairwise mutual information. As the robot performs exploratory motions based on motor babbling we capture and study the time-varying changes in the relationships. We provide analysis of the instantaneous and average sharing of information and extrapolate the meaning in relation to the physical properties of the robot's body. Results from a simulated planar system validate the use of mutual information as a tool not only for the analysis of the relationships between the sensorimotor signals but also to drive exploratory motions.
\TODO
\end{abstract}
% =============================================================================
%                                                                             |
%                                                                             |
% ------------------------------- SECTION ------------------------------------|
%                                                                             |
%                                                                             |
% =============================================================================
\section{Introduction}\label{sec:intro}
\TODO
\begin{figure}[!th]
	\centering
	\includegraphics[width=0.9\columnwidth]{fig/general_overview.png}
	\caption{General overview.}
	\label{fig:general_overview}
\end{figure}


% SUBSECTION ==================================================================
\subsection{Related works}

Refer to Mannella, Gama, Marcel, Kanazawa, Husbands, Hoffmann (puppy).

% SUBSECTION ==================================================================
\subsection{Contributions}
\TODO
In this work we present an analysis based on the information theory of the dnyamic relationships exsiting among the sensorimotor signals of a robot. In the analysis we show how the amount of information sharing in the system is related to the motion of the robot and the tactile events that occur duting motion. Furthermore, using nonnegative matrix factoriztion we identify operation modes of the robot based only of the mutual information measurments. In a case study we demonstrate how by only foucsing on the mutual information and without having knowledge about the morphology of the robot, an excitation trajectory can be devised that avoids self contact. A comparison of this trajectory against conventional trajevetory design methods is presented.

% =============================================================================
%                                                                             |
%                                                                             |
% ------------------------------- SECTION ------------------------------------|
%                                                                             |
%                                                                             |
% =============================================================================
\section{The robot model}
\TODO

\begin{figure}[!th]
	\centering
	\includegraphics[width=0.9\columnwidth]{fig/extended_dual_arm_robot.pdf}
	\caption{The planar dual arm robot.}
	\label{fig:extended_dual_arm_robot}
\end{figure}

% =============================================================================
%                                                                             |
%                                                                             |
% ------------------------------- SECTION ------------------------------------|
%                                                                             |
%                                                                             |
% =============================================================================
\section{Dynamic functional connectivity}

Many metrics based on information theory have been proposed to evaluate sensorimotor relationships \cite{Bonsignorio2020EntropyBasedMetrics}

DFC has has mainly being used to study functional networks in the brain. For instance, in \cite{Christiaen2020Dynamicfunctionalconnectivity}, it was used to show states of lower functional connectivity during onsets of epilepsy.


The work in \cite{Zhou2020Earlychildhooddevelopmental} used NNMF to \textit{In this study, a parts-based network-decomposition technique, known as non-negative matrix factorization (NMF), was applied to identify a set of possible abnormal connectivity patterns in brains affected by ASD, using resting-state electroencephalographic (EEG) data.}
 
The most common approach to express the functional connectivity is via a sliding time window \cite{Preti2017dynamicfunctionalconnectome}
\TODO

\begin{figure}[!th]
	\centering
	\includegraphics[width=0.9\columnwidth]{fig/nnmf.png}
	\caption{Decomposition of the mutual information data matrix using non negative matrix factorization.}
	\label{fig:nnmf}
\end{figure}

\textbf{Rephrase:} \textit{To extract information from these connectivity graphs, the NMF \cite{Fu2019Nonnegativematrixfactorization} technique was used. NMF is an unsupervised machine learning algorithm that can be used to decompose an input matrix A into two components: a matrix W of bases, or subgraphs, and a matrix H of their corresponding contributions to the input matrix A (39). Connectivity graphs were combined to serve as the input matrix (A) for NMF that yielded subgraphs (W), taken to be functional resting-state alpha networks, and associated time series of contributions (H), taken to be the relative temporal activation of each of these networks.}

NNMF has also being used to detect communities in networks \cite{Wang2011Communitydiscoveryusing,Luo2021Symmetricnonnegativematrix}


Other methods to detect repeating patterns in DFC include the cosine similarity \cite{Menon2019comparisonstaticdynamic}

There are various methods to select an adequate number of factors \cite{Muzzarelli2019RankSelectionNon}. To choose the number of factors we followed as in \cite{Phalen2020Nonnegativematrix} and "which was chosen by performing NMF iteratively for increasing values of k and selecting the inflection point, or elbow, in the reconstruction residual sum of squares
error curve"

In \cite{Stiso2020Learningbraincomputer} it is shown how the basis graph change their expressing during learning of a ask


To show graphically how the different factors cluster naturally depending on the touched regions we used the PaCMAP dimensionality reduction method \cite{Wang2021Understandinghowdimension} given its properties to preserve aspects of the global and local structure when reducing into the latent space \cite{Huang2022Towardscomprehensiveevaluation}.
% =============================================================================
%                                                                             |
%                                                                             |
% ------------------------------- SECTION ------------------------------------|
%                                                                             |
%                                                                             |
% =============================================================================
\section{Simulation results}
\TODO

% =============================================================================
%                                                                             |
%                                                                             |
% ------------------------------- SECTION ------------------------------------|
%                                                                             |
%                                                                             |
% =============================================================================
\section{Case study: robot excitation trajectories}
\TODO
In this section we use the instantaneous mutual information to generate trajectories for the left and right arms avoiding potential collisions. This is done agnostic to the actual morphology of the robot. In contrast we use a standard method for the design of excitation trajectories and compare the results.

% =============================================================================
%                                                                             |
%                                                                             |
% ------------------------------- SECTION ------------------------------------|
%                                                                             |
%                                                                             |
% =============================================================================
\section{Beyond robotics}
\TODO

% =============================================================================
%                                                                             |
%                                                                             |
% ------------------------------- SECTION ------------------------------------|
%                                                                             |
%                                                                             |
% =============================================================================
\section{Conclusions}\label{sec:conclusion}



\printbibliography 
\end{document}